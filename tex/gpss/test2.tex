% test with
% pandoc -R -f latex test2.tex -F ../../myfilter.py -t markdown -o test2.md

%\documentclass[serif, mathserif, professionalfont]{beamer}
\usetheme{BlueGrey}
\usenavigationsymbolstemplate{}

\setlength{\parskip}{1em}

% package includes

% Bibliography
\usepackage{natbib}

% maths
\usepackage{amsmath}
\usepackage{amssymb}

% fonts
\usepackage{bm}
\usepackage{pxfonts}
\usepackage{palatino}

% programming
\usepackage{listings}

% multimedia and drawing
\usepackage{multimedia}
\usepackage{xmpmulti}
\usepackage{graphicx}
\usepackage{subfigure}
\usepackage{tikz}
\usepackage{media9}
\usepackage{animate}
\usepackage{pgf}

% layout
\usepackage{relsize}
\usepackage{setspace}
\usepackage{multirow}
\usepackage{textpos}

\input{notation_def.tex}
\input{definitions.tex}
\input{graphicalModels.tex}
\input{people.tex}




\newcommand{\inputVals}{\inputVector}
%\global\long\def\blackBackground{}
\newcommand{\rbfKernel}{EQ}
\newcommand{\rbfKernelLong}{exponentiated quadratic}

\begin{document}

\title[Gaussian Processes]{Introduction to Gaussian Processes}

\author[Lawrence]{Neil D. Lawrence}
\titlegraphic{\href{http://gpss.cc}{\includegraphics[width=2cm]{../../../gp/tex/diagrams/logo}}}

\date[MLSS 2016]{MLSS, Arequipa, Peru\\
2nd August 2016}

\frame{\maketitle}
\begin{frame}
  \frametitle{Outline}

  \tableofcontents[hideallsubsections] 

\end{frame}
\newsection{Gaussian Processes}{../../../gp/tex/talks/gpbook.tex}
\includetalkfile{../../../gp/tex/talks/overdeterminedGp.tex}
\includetalkfile{../../../gp/tex/talks/underdeterminedGp.tex}
\begin{frame}
  \frametitle{Gaussian Process}
  \[
  \dataScalar_i(\inputVector_i) = \mappingFunction(\inputVector_i) + \noiseScalar_i
  \]
  \begin{itemize}
  \item Place a prior over the process as well as the noise.
  \item Leads to models that are not i.i.d.
    \item Contrast with classical model's objective function:
    \[
    \sum_{i=1}^n (1 - y_i(\mappingVector^\top\inputVector_i - b))_{+} + \lambda \mappingVector^\top\mappingVector
    \]
  \end{itemize}
\end{frame}
\begin{frame}
  \frametitle{Model and Algorithm}
  \begin{itemize}
  \item I'm keen on the idea of a conceptual separation model and algorithm.
  \item Model is how you encode the regularities of the universe.
  \item Algorithm is how you combine that model with data.
    \[
    \text{data} + \text{model} \rightarrow \text{prediction}
    \]
    \item Of course often we are restricted in modeling choice due to lack of algorithms.
  \end{itemize}
\end{frame}
    

\newsubsection{Distributions over Functions}{../../../gp/tex/talks/gp_extremely_short.tex}
\includetalkfile{../../../gp/tex/talks/gpdistfunc}
%\includetalkfile{../../../gp/tex/talks/gpTwoPointIntro}
\newsubsection{Two Point Marginals}{../../../gp/tex/talks/gptwopointpred}

\includetalkfile{../../../kern/tex/talks/rbfcovariance}
\includetalkfile{../../../kern/tex/talks/computingRbfCovariance}

\includetalkfile{../../../gp/tex/talks/gpInterpolation.tex}
\begin{frame}
  \frametitle{Gaussian Noise}
  \begin{itemize}
  \item Gaussian noise model, 
    \[
    p\left(\dataScalar_{i}|\mappingFunction_{i}\right) = \gaussianDist{\dataScalar_{i}}{\mappingFunction_{i}}{\dataStd^2}
    \]
    where $\dataStd^2$ is the variance of the noise.
    \item Equivalent to a covariance function of the form
    \[
    \kernelScalar(\inputVals_i, \inputVals_j) = \delta_{i, j}\dataStd^2
    \]
    where $\delta_{i,j}$ is the Kronecker delta function.
    \item Additive nature of Gaussians means we can simply add this term to existing covariance matrices.                       
  \end{itemize}
\end{frame}

\newsubsection{GP Regression}{../../../gp/tex/talks/gpregression.tex}
\includetalkfile{../../../ml/tex/talks/olympicMarathonGpFit.tex}
\newsection{GP Non-Gaussian}{../../../gp/tex/talks/gpNoiseModels.tex}

\includetalkfile{../../../gp/tex/talks/gpoptimize.tex}



\newsection{Parametric Models are a Bottleneck}{../../../gp/tex/talks/bottleneck.tex}


\includetalkfile{../../../gp/tex/talks/mercer.tex}


% \includetalkfile{../../../kern/tex/talks/rbfcovariance}


\newsubsection{Covariance Functions}{../../../gp/tex/talks/basisFunctions.tex}
%\includetalkfile{../../../gp/tex/talks/multivariateGaussianProperties.tex}
\includetalkfile{../../../kern/tex/talks/rbfbasiscovariance.tex}

\newsubsection{Multivariate Gaussian Properties}{../../../gp/tex/talks/multivariateGaussianProperties.tex}
\newsubsection{An Infinite Basis}{../../../gp/tex/talks/infiniteBasis.tex}
\includetalkfile{../../../kern/tex/talks/rbfbasiscovariance.tex}
\includetalkfile{../../../kern/tex/talks/rbfcovariance.tex}
\includetalkfile{../../../kern/tex/talks/mlpcovariance.tex}
\newsubsection{Constructing Covariance Functions}{../../../gp/tex/talks/constructingCovariance}

\newsubsection{Bochner's Theorem}{../../../gp/tex/talks/bochnerTheorem.tex}
\includetalkfile{../../../kern/tex/talks/oucovariance.tex}
\includetalkfile{../../../kern/tex/talks/matern32covariance.tex}
\includetalkfile{../../../kern/tex/talks/matern52covariance.tex}
\includetalkfile{../../../kern/tex/talks/rbfcovariance.tex}

%\includetalkfile{../../../kern/tex/talks/covfuncSamples}
\includetalkfile{../../../kern/tex/talks/mlpcovariance}
\includetalkfile{../../../kern/tex/talks/lincovariance}


\newsection{GP Limitations}{../../../gp/tex/talks/gpLimitations}
%\newsubsection{Gaussian Process Review}{../../../gp/tex/talks/gpreview_verylong

% \section{Conclusions}


% % \subsection{Summary}

% % \begin{frame}
% %   \frametitle{Summary}
% %   \begin{itemize}
% %   \item Broad introduction to Gaussian processes.
% %     \begin{itemize}
% %     \item Started with Gaussian distribution.
% %     \item Motivated Gaussian processes through the multivariate density.
% %     \end{itemize}
% %   \item Emphasized the role of the covariance (not the mean).
% %   \item Performs nonlinear regression with error bars.
% %   \item Parameters of the covariance function (kernel) are easily
% %     optimized with maximum likelihood.
% %   \end{itemize}
% % \end{frame}

\newsection{Kalman Filter}{../../../kern/tex/talks/markovDerivation}
\includetalkfile{../../../kern/tex/talks/markovcovariance}
\includetalkfile{../../../kern/tex/talks/markovprecision}
\includetalkfile{../../../kern/tex/talks/rbfcovariance}
\includetalkfile{../../../kern/tex/talks/rbfprecision}
\includetalkfile{../../../kern/tex/talks/markovprecision}
\includetalkfile{../../../gp/tex/talks/gpKalmanFilterKronecker}
\includetalkfile{../../../multigp/tex/talks/gpKalmanToMultiTask}



\newsection{Dimensionality Reduction}{../../../dimred/tex/talks/sixexample.tex}
\newsubsection{Existing Methodologies}{../../../gplvm/tex/talks/existingMethodologies.tex}
\includetalkfile{../../../gplvm/tex/talks/notation.tex}
\newsubsection{Dual Probabilistic PCA}{../../../gplvm/tex/talks/dppca.tex}

\newcommand{\inputVals}{\latentVector}
%\newsubsection{Gaussian Processes}{../../../gp/tex/talks/gpdistfunc.tex}
%\input{../../../gp/tex/talks/covfunctions.tex}
\newsubsection{Nonlinear Latent Variable Models}{../../../gplvm/tex/talks/nonlinearLatent.tex}
\newsubsection{Examples}{../../../gplvm/tex/talks/exampleList.tex}
\includetalkfile{../../../gplvm/tex/talks/latentDoodleSpace.tex}
\includetalkfile{../../../gplvm/tex/talks/gaussianFace.tex}
\includetalkfile{../../../gplvm/tex/talks/characterControl.tex}

\newcommand{\returnLabel}{extensionsReturn}
\begin{frame}[label=\returnLabel]
  \frametitle{Other Topics}
  \begin{itemize}
  \item Local distance preservation \hyperlink{backconstraints}{\beamergotobutton{Details}}
  \item Dynamical models \hyperlink{dynamics}{\beamergotobutton{Details}}
  \item Hierarchical models \hyperlink{hierarchies}{\beamergotobutton{Details}}
  \item Bayesian GP-LVM \hyperlink{bayesGplvm}{\beamergotobutton{Details}}
  \end{itemize}
\end{frame}
\newsubsection{Back Constraints}{../../../gplvm/tex/talks/backConstraints.tex}
\newsubsection{Dynamics}{../../../gplvm/tex/talks/dynamics.tex}
\newsubsection{Hierarchical GP-LVM}{../../../gplvm/tex/talks/hgplvm.tex}
\newsubsection{Bayesian GP-LVM}{../../../gplvm/tex/talks/bayesGplvmIntro.tex}
\includetalkfile{../../../gplvm/tex/talks/variationalBayesGPLVM_long.tex}
\includetalkfile{../../../gplvm/tex/talks/ardGplvm.tex}
\newsubsection{Gaussian Process Dynamical Systems}{../../../gplvm/tex/talks/gpds.tex}
\newsubsection{Shared GP-LVM}{../../../gplvm/tex/talks/mrdGplvm.tex}



\begin{frame}[allowframebreaks]
  \frametitle{References}

  {\footnotesize \bibliographystyle{pdf_abbrvnat}
    \bibliography{lawrence,other,zbooks}
  }


\end{frame}

\appendix

\end{document}
